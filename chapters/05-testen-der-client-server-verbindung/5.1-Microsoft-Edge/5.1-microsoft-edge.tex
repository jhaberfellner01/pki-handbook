\section{Microsoft Edge}

\subsection{Testumgebung}

Für den Testaufbau wurde ein Windows-System verwendet, auf dem Nginx installiert und als Webserver eingerichtet wurde. 
Die Webseite wurde lokal gehostet, sowie auf demselben Rechner aufgerufen.
Der Fokus lag darauf, den Server mit einem eigenen Zertifikat auszustatten und anschließend eine vollständig vertrauenswürdige HTTPS-Verbindung herzustellen.

\begin{itemize}
    \item \textbf{Server:} Windows 11 (Nginx, IP: 192.168.3.150)
    \item \textbf{Client:} Windows 11 (Chrome/Edge)
    \item \textbf{Protokoll:} HTTPS (Port 443)
    \item \textbf{Zertifikate:}
    \begin{itemize}
        \item Root CA: \texttt{root\_CA.crt}
        \item Intermediate CA: \texttt{intermediate\_CA.crt}
        \item Server-Zertifikat: \texttt{server.crt}
    \end{itemize}
\end{itemize}

\subsection{Import der Zertifikate unter Windows}
Damit der Browser dem Serverzertifikat vertraut, müssen die Root CA und optional die Intermediate CA 
im Windows-Zertifikatsspeicher bzw. für macOS in der Apple Keychain hinterlegt werden.  
Dies ist notwendig, da das Zertifikat selbstsigniert bzw. von einer eigenen CA ausgestellt wurde und
Microsoft Edge auf den Betriebssystemzertifikatsspeicher zurückgreift.

\begin{enumerate}
    \item \textit{Programme → Dienstprogramme → Schlüsselbundverwaltung} öffnen
    \item In der linken Spalte den Bereich \textit{System} auswählen
    \item Die Datei Certficate Datei per Drag-and-Drop in Apple Keychain ziehen
    \item Das Zertifikat öffnen und unter \textit{Trust} die Option
          \textit{When using this certificate: Always Trust} setzen
\end{enumerate}

\begin{figure}[H]
    \centering
    \includegraphics[width=0.8\linewidth]{apple-keychain-access-cert-import.png}
    \caption{Import der Root~CA in die Appley Keychain}
    \label{fig:apple-cert-import}
\end{figure}

\subsection{Verbindungstest}

Nach Abschluss des Imports wurden der Browser neu gestartet und anschließend die Webseite erneut aufgerufen.
Die Verbindung wurde nun als vollständig sicher angezeigt, da der Browser die Zertifikatskette korrekt validieren konnte.

\begin{figure}[H]
    \centering
    \includegraphics[width=0.8\linewidth]{edge-certificate-correct.png}
    \caption{Zertifikat in Microsoft Edge als sicher erkannt}
    \label{fig:edge-cert-import-working}
\end{figure}