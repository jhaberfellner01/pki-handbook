\subsubsection*{Verhalten von Microsoft Edge}

Da Microsoft Edge seit der Neuausrichtung 2020 vollständig auf der Chromium-Engine basiert, 
übernimmt der Browser das gleiche Widerrufsmodell wie Google Chrome. Klassische PKI-Mechanismen – 
insbesondere CRLs über CDP-Einträge und OCSP-Abfragen über das AIA-Feld – werden auch von Edge 
nicht ausgewertet. Die Gründe entsprechen im Wesentlichen denen von Chrome:

\begin{itemize}
\item \textbf{Performance}: Edge führt keine OCSP-Abfragen durch und lädt keine CRLs, um Netzwerkverzögerungen zu vermeiden.
\item \textbf{Stabilität}: Microsoft folgt der Chromium-Architektur, die große CRLs und potenzielle Zeitüberschreitungen als unzuverlässig einstuft.
\item \textbf{Datenschutz}: Wie Chrome würde auch Edge durch OCSP jede aufgerufene Website gegenüber dem OCSP-Responder offenlegen.
\end{itemize}

Damit ignoriert Edge – trotz korrekt gesetzter Felder im Zertifikat – sowohl CRL Distribution Points als auch OCSP-Responder-URLs und ersetzt die klassischen PKI-Widerrufsmechanismen durch Chromium-eigene Strukturen.

\paragraph*{CRLSets in Microsoft Edge}

Microsoft Edge bezieht dieselben \textit{CRLSets}, die Google für Chromium bereitstellt. Die Funktionsweise unterscheidet sich nicht:

\begin{itemize}
\item Die Datenbasis stammt ausschließlich von Google.
\item Nur sicherheitsrelevante, global bedeutende Widerrufe werden aufgenommen (z.,B. kompromittierte CAs).
\item End-Entity-Zertifikate werden praktisch nie berücksichtigt.
\item Private PKIs werden grundsätzlich ignoriert.
\end{itemize}

Edge erweitert CRLSets nicht durch eigene Widerrufslisten und betreibt auch keine eigene Infrastruktur für OCSP-Stapling oder CRL-Verteilung für interne Zertifizierungsstellen.

\subparagraph*{Einschränkungen in Edge}

Die Einschränkungen entsprechen denen von Chrome:

\begin{itemize}
\item \textbf{Keine Unterstützung privater CRLs}: Edge ruft selbst veröffentlichte CRLs nicht ab.
\item \textbf{Keine OCSP-Nutzung}: Auch ein intern erreichbarer OCSP-Responder wird nicht abgefragt.
\item \textbf{Ignorieren privater CAs}: CRLSets enthalten ausschließlich Widerrufe aus dem öffentlichen WebPKI-Ökosystem.
\item \textbf{Widerrufe werden nicht erkannt}: Interne, widerrufene Serverzertifikate werden weiterhin als gültig angezeigt.
\end{itemize}

\subsubsection*{Problemstellung: Private CA und Widerruf in Microsoft Edge}

Durch die vollständige Übernahme des Chromium-Sicherheitsmodells ergeben sich für Edge im Laboraufbau exakt dieselben Probleme wie bei Chrome:

\begin{itemize}
\item Die im Zertifikat hinterlegte CRL wird nicht abgerufen.
\item OCSP findet nicht statt, selbst wenn ein interner Responder korrekt bereitsteht.
\item Widerrufe in einer privaten PKI werden grundsätzlich nicht berücksichtigt.
\item Edge zeigt widerrufene Zertifikate weiterhin als vertrauenswürdig an.
\end{itemize}